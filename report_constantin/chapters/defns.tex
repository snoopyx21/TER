\newglossaryentry{LAN}
{
	name={\textit{Local Area Network}},
	description={Réseau informatique restreint où les différents acteurs s'échangent les données au niveau de la couche liaison, c'est-à-dire sans accèder à Internet.}
}

\newglossaryentry{VM hopping}{
	name={\textit{VM hopping}},
	description={Attaque visant à prendre le contrôle d'une machine virtuelle et par la suite, tenter de prendre le contrôle d'une autre. Si plusieurs machines virtuelles sont présentes sur le même hôte, alors elles sont des cibles de l'attaque.}
}

\newglossaryentry{RBAC}{
	name={\textit{Role-Based Access Control}},
	description={concept de sécurité du réseau selon lequel le réseau accorde des droits aux utilisateurs en fonction de leur rôle dans l’entreprise}
}

\newglossaryentry{Black Hats}{
	name={\textit{Black Hats}},
	description={Spécialiste en informatique qui recherche différents moyens de contourner la sécurité mise en place au niveau logiciel ou matériel dans le but de nuire à autrui, faire du profit ou encore obtenir des informations. La plupart des actions effectués par ces spécialistes sont illégales.}
}

\newglossaryentry{TCP/IP}{
	name={TCP/IP},
	description={Architecture réseau en 4 couches dans laquelle les protocoles TCP et IP jouent un rôle prédominant.}
}

\newglossaryentry{TPA}{
	name={\textit{Trusted third party}},
	description={Entité facilitant les interactions entre deux parties qui ont tous les deux confiance en cette entité.}
}

\newglossaryentry{VMs}{
	name={\textit{Virtual Machine}},
	description={Une machine virtuel (Virtual Machine en anglais, abrégé VM) est un environnement d'applications ou un système d'epxloitation installé par un logiciel ou instancée par un hyperviseur permettant l'émulation d'un matériel dédié.}
}

\newglossaryentry{hyperviseur}{
	name={Hyperviseur},
	description={Outil permettant à plusieurs systèmes d'exploitation de tourner sur la même machine physique en même temps.}
}

\newglossaryentry{firewalls}{
	name=firewalls,
	description={Un pare-feu (ou firewall en anglais) est un ensemble de composants placé entre deux réseaux filtrant l'ensemble du trafic provenant du réseau externe vers le réseau interne et réciproquement. De plus, en fonction de la politique de sécurité mise en place, un firewall filtre également le trafic non autorisé et il doit être impénétrable et inattaquable. \cite{firewall} }
}
\newglossaryentry{IDS}{%
	name=IDS,
	description={Un \textit{Intrusion Detection System} est un dispotif ou une application qui surveille en temps réel un réseau ou un(des) système(s) d'activités malicieuses ou d'une violation de la politique de sécurité mise en place. Équipé d'un système d'alarme, ils ont pour but de prévenir l'utilisateur des activités malicieuses et d'offrir des informations sur ces dernières via un système de log. }
}

\newglossaryentry{IPS}{
	name=IPS,
	description={Un \textit{Intrusion Prevention System} est un dispotif ou une application ayant les mêmes fonctionnalités qu'un IDS, mais peut mettre en place des contre-mesures afin de stopper une activité malveillante dans un délai court. }
}

\newglossaryentry{prises d'empreintes}{
	name=prise d'empreinte,
	description={Une prise d'empreinte de la pile TCP/IP est un procédé permettant de déterminer l'identité du système d'exploitation utilisé sur une machine distante en analysant les paquets provenant de cet hôte. \cite{prise_empreinte_def}}
}

\newglossaryentry{OpenFlow}{
	name=OpenFlow,
	description={OpenFlow est un protocole de communication qui propose une architetcure Software-defined networking (SDN).
	}
}

\newglossaryentry{XenServer}{
	name=XenServer,
	description={XenServer est un logiciel libre de virtualisation, plus précisément un hyperviseur de machine virtuelle.
	}
}

\newglossaryentry{NHRP}{
	name=\textit{Next Hop Resolution Protocol},
	description={Protocole permettant à une source d'envoyer des données à une destination en utilisant et apprenant les routes les plus directes permettant de joindre cette destination. \cite{nhrp}}
}

\newglossaryentry{GRE}{
	name=\textit{Generic Routing Encapsulation},
	description={Protocole de mise en tunnel entre deux réseaux qui permet d'encapsuler n'importe quel paquet de la couche réseau. \cite{gre}}
}

\newglossaryentry{POX}{
	name={POX},
	description={POX is an open source development platform for Python-based software-defined networking (SDN) control applications, such as OpenFlow SDN controllers. \cite{pox}}
}

\newglossaryentry{XACML}{
	name={\textit{eXtensible Access Control Markup Language} (XACML)},
	description={Standard pour définir les contrôles d'accès et les autorisations.}
}

\newglossaryentry{SLA}{
	name={\textit{Service Level Agreement}},
	description={Contrat définissant une qualité de service, prestation prescrite entre un fournisseur de service et un client.}
}